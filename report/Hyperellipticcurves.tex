%TODO
%Example for why helliptics aren't grous 
%Example principal divisors
%Example of why divisors are so hard to represent
%Definition of semi-reduced divisor, weight of a divisor
%Why does cantors algorithm work and example 
%Finding points 
%Counting points 

Unlike elliptic curves, when the genus $\mf{g}$ of a curve $\mf{C} $ is greater than $1$, the set of points on $\mf{C}$ will not always form a group. 

% \begin{example}

% \end{example}

\subsection{The Jacobian of a Hyperelliptic Curve}

Luckily, there is another way to form an abelian group with hyperelliptic curves. Indeed, let $\mf{D}$ be the set of all formal finite sums 

$$ \sum_i m_i P_i $$ 

where $m_i \in \mathbb{Z}$ and $P_i$ are points on the curve $\mf{C}$. We call elements of $\mf{D}$ divisors of $\mf{C}$. Given a rational function $f$ in $\mathbb{Z}_p[\mf{C}]$, we can define the corresponding divisor to $f$ as

$$(f) = \sum_i m_i P_i $$ where $P_i$ are the zeros and poles of $f$ with multiplicities $m_i$. 

% \begin{example}
% \end{example} 

Divisors of this form are called principal divisors and we let $\mf{P}$ denote the subset of all of them in $\mf{D}$. If we define the operation on $\mf{D}$ by 

$$ \sum_i m_i P_i  + \sum_i m^\prime_i P_i  = \sum_i (m_i+m^\prime) P_i $$ 

then $\mf{D}$ becomes and abelian group. Unfortunetly, this group is far too large and unstructured for cryptographic purposes. So we consider the subgroup $\mf{D}^0$ of all divisors of $\mf{D}$ whose coefficients sum to $0$. That is, divisors $ \sum_i m_i P_i $ such that $\sum_i m_i = 0$. \\ 

This subgroup is still infinite, but that can be remedied by defining two divisors $D_1, D_2$ of $\mf{D}^0$ to be equal if $D_1 - D_2$ is equal to the divisor of a rational function on $\mf{C}$. That is, $D_1 - D_2 = (f) $ for $f \in \mathbb{Z}_p[\mf{C}]$. This new quotient group, denoted $$\mf{J} = \mf{D}^0 / \mf{P}$$ is called the jacobian of the curve $\mf{C}$ and is a finite cyclic group. This will be the group used to build hyperelliptic cryptosystems.

\subsection{Representation of Divisors}

Athough the Jacobian $\mf{J}$ of an hyperelliptic curve $\mf{C}$ is a finite abelian group, elements of $\mf{J}$ are very hard to represent. 

% \begin{example}
% \end{example} 

% Definition of semi-reduced divisor and how to find it and talk about the weight of a divisor y1report.pdf 

To make the group operation in $\mf{J}$ tractable, we ustilize the mumford representation of a divisor which is described as follows. Let $D$ be a semi-reduced with points $P_i = (x_i,y_i)$. We associate to $D$ polynomials $a,b \in \mathbb{Z}_p[x]$ such that $$a(x) = \prod^r_i (x - x_i) $$ $$ b(x_i) = y_i \text{ } 1 \leq i \leq r $$ where $\deg b < \deg a$ and $(x - x_i)^{k_i} \mid b - y_i$, if $k_i$ is the multiplicity of $P_i$. Denote this representation $D \stackrel{\text{def}}{=} \text{div} (a,b)$.

\subsection{The Group Operation}

The group operation can be divided into two parts - \textit{Composition} and \textit{reduction} as described in [\ref{TanjaLange}]. \\ 	

Given two divisors represented as $D_1= \text{div}(a_1,b_1), D_2 = \text{div} (a_2,b_2) $ 

\begin{enumerate}[1.]
	\item compute $d_0 = \text{gcd}(a_1,a_2)$ and find the unique $c_1,e_1 \in \mathbb{Z}_p[x]$ such that $d_0 = c_1a_1 + e_1a_2$ 
	\item compute $d = \text{gcd}(d_1,b_1 + b_2)$ and find the unique $c_2, e_2 \in \mathbb{Z}_p[x]$ such that $d = c_2d_1 + e_2(b_1 + b_2)$ 
	\item compute $a_3 = \frac{a_1,a_1}{d^2}$
	\item compute $b_3 = \frac{c_2c_1a_1 + c_2e_1a_2 + e_2(b_1b_2 + f)}{d} \text{ mod } \frac{a_1,a_1}{d^2}$
	\item \label{repeat} compute $a_3^\prime = \frac{f - b_3^2}{a_3}$ and $b_3^\prime = - b_3$ mod $a_3^\prime$ 
	\item while $\deg (a_3^\prime ) > g$, reassign $a_3 = a_3^\prime, b_3 = b_3^\prime$ and repeat step \ref{repeat}
	\item divide $a_3^\prime $ by its leading coefficient so that $a_3^\prime $ becomes monic
	\item the output $div(a_3^\prime,b_3^\prime) = D_1 + D_2$ 
\end{enumerate}


% Why Does this work? 

% \begin{example}
% \end{example}

\subsection{Scalar Multiplication}
\subsection{Counting Points}
\subsection{Parameter Selection}


