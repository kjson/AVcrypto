\subsection{The Jacobian of a Hyperelliptic Curve}

Unlike the case of elliptic curves, when the genus $\mathfrak{g}$ of a curve $\mathfrak{C} $ is greater than 1, the set of points on $\mathfrak{C}$ will not always form a group. 

\begin{example}

\end{example}

Luckily, there is another way to form an abelian group with hyperelliptic curves. Indeed, let $\mathfrak{D}$ be the set of all formal finite sums 

$$ \sum_i m_i P_i $$ 

where $m_i \in \mathbb{Z}$ and $P_i$ are points on the curve $\mathfrak{C}$. We call elements of $\mathfrak{D}$ divisors of $\mathfrak{C}$. Given a rational function $f$ in $\mathbb{Z}_p[\mathfrak{C}]$, we can define the corresponding divisor to $f$ as

$$(f) = \sum_i m_i P_i $$ where $P_i$ are the zeros and poles of $f$ with multiplicities $m_i$. Divisors of this form are called principal divisors and we let $\mathfrak{P}$ denote the subset of all of them in $\mathfrak{D}$. If we define the operation on $\mathfrak{D}$ by 

$$ \sum_i m_i P_i  + \sum_i m^\prime_i P_i  = \sum_i (m_i+m^\prime) P_i $$ 

then $\mathfrak{D}$ becomes and abelian group. Unfortunetly, this group is far too large for cryptographic purposes. So we consider the subgroup $\mathfrak{D}^0$ of all divisors of $\mathfrak{D}$ whos coefficients sum to $0$. That is, divisors $ \sum_i m_i P_i $ such that $\sum_i m_i = 0$. \\ 

Even though this subgroup is still infinite, we can define two divisors $D_1, D_2$ of $\mathfrak{D}^0$ to be equal if $D_1 - D_2$ is equal to the divisor of a rational function on $\mathfrak{C}$. That is, $D_1 - D_2 = (f) $ for $f \in \mathbb{Z}_p[\mathfrak{C}]$. This new quotient group, denoted $$ \mathfrak{J} = \mathfrak{D}^0 / \mathfrak{P}$$ is called the jacobian of the curve $\mathfrak{C}$ and is a finite abelian group. This will be the group used to build hyperelliptic cryptosystems.

\subsection{Representation of Divisors}

Athough the Jacobian $\mathfrak{J}$ of an hyperelliptic curve $\mathfrak{C}$ is a finite abelian group, elements of $\mathfrak{J}$ are very hard to represent. 

\begin{example}
\end{example} 

