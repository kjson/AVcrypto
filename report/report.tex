\documentclass[10pt,a4paper]{article}
\usepackage{
  amsmath,amsfonts,euscript,enumerate,mathrsfs,
  hyperref,amsthm,amssymb,upref,graphics,color
}

\usepackage[all]{xy}
\hypersetup{
  colorlinks,breaklinks,urlcolor=blue,linkcolor=black
  }

\theoremstyle{plain} 
  \newtheorem{theorem}[subsection]{Theorem}
  \newtheorem*{thmnonumber}{Theorem}
  \newtheorem{proposition}[subsection]{Proposition}
  \newtheorem{lemma}[subsection]{Lemma}
  \newtheorem*{lemmanonumber}{Lemma}
  \newtheorem{corollary}[subsection]{Corollary}
  \newtheorem*{cornonumber}{Corollary}
  \newtheorem{nothing}[subsection]{}
  \newtheorem{sheafificationThm}[subsection]{Sheafification Theorem}
  \newtheorem{subtheorem}{Theorem}[subsection]
  \newtheorem{subproposition}[subtheorem]{Proposition}
  \newtheorem{sublemma}[subtheorem]{Lemma}
  \newtheorem{subcorollary}[subtheorem]{Corollary}
  \newtheorem{subnothing}[subtheorem]{}

\theoremstyle{definition}
  \newtheorem{definition}[subsection]{Definition}
  \newtheorem{definitions}[subsection]{Definitions}
  \newtheorem*{definonumber}{Definition}
  \newtheorem{nothing*}[subsection]{}
  \newtheorem{example}[subsection]{Example}
  \newtheorem{examples}[subsection]{Examples}
  \newtheorem*{solution}{Solution}
  \newtheorem*{exnonumber}{Example}
  \newtheorem*{quesnonumber}{Question}
  \newtheorem{question}{Question}
  \newtheorem{problem}[subsection]{Problem} 
  \newtheorem{exercise}[subsection]{Exercise} 
  \newtheorem{notation}[subsection]{Notation}
  \newtheorem{notations}[subsection]{Notations}
  \newtheorem{step}{Step}
  \newtheorem*{claim}{Claim}
  \newtheorem{assumptions}[subsection]{Assumptions}
  \newtheorem{subdefinition}[subtheorem]{Definition}
  \newtheorem{subnotation}[subtheorem]{Notation}
  \newtheorem{subnotations}[subtheorem]{Notations}
  \newtheorem{subexample}[subtheorem]{Example}
  \newtheorem{subnothing*}[subtheorem]{}

\theoremstyle{remark}
  \newtheorem*{remark}{Remark}
  \newtheorem*{remarks}{Remarks}
  \newtheorem*{warning}{Warning}
  \newtheorem*{smallexample}{Example}


\newcommand{\coker}{    \oVperatorname{{\rm coker}}}
\newcommand{\Aut}{    \oVperatorname{{\rm Aut}}}
\newcommand{\Spec}{   \oVperatorname{{\rm Spec}}}
\newcommand{\Proj}{   \oVperatorname{{\rm Proj}}}
\newcommand{\rank}{   \operatorname{{\rm rank}}}
\newcommand{\Reg}{    \operatorname{{\rm Reg}}}
\newcommand{\haut}{   \operatorname{{\rm ht}}}
\newcommand{\supp}{   \operatorname{{\rm supp}}}
\newcommand{\image}{  \operatorname{{\rm im}}}
\newcommand{\ord}{    \operatorname{{\rm ord}}}
\newcommand{\bideg}{  \operatorname{{\rm bideg}}}
\newcommand{\trdeg}{  \operatorname{{\rm trdeg}}}
\newcommand{\Frac}{   \operatorname{{\rm Frac}}}
\newcommand{\Char}{   \operatorname{{\rm char}}}
\newcommand{\Span}{   \operatorname{{\rm Span}}}
\newcommand{\Sing}{   \operatorname{{\rm Sing}}}
\newcommand{\Pic}{    \operatorname{{\rm Pic}}}
\newcommand{\Cl}{   \operatorname{{\rm Cl}}}
\newcommand{\dom}{    \operatorname{{\rm dom}}}
\newcommand{\codom}{  \operatorname{{\rm codom}}}
\renewcommand{\div}{  \operatorname{{\rm div}}}
\newcommand{\id}{   \operatorname{{\rm id}}}
\newcommand{\ob}{   \operatorname{{\rm ob}}}
\newcommand{\Hom}{    \operatorname{{\rm Hom}}}
\newcommand{\Set}{    \operatorname{{\rm\bf Set}}}
\newcommand{\Top}{    \operatorname{{\rm\bf Top}}}
\newcommand{\Grp}{    \operatorname{{\rm\bf Grp}}}
\newcommand{\CRng}{   \operatorname{{\rm\bf CRng}}}

\newcommand{\Mod}[1]{\mbox{\rm ${#1}$-\bf Mod}}
\newcommand{\setspec}[2]{\big\{\,#1\, \mid \,#2\, \big\}}
\newcommand{\powerset}{\raisebox{\depth}{\Large $\wp$}}

\newcommand{\notdiv}{\not\hspace{\mylength}\mid}
\newcommand{\epi}{\twoheadrightarrow}
\newcommand{\overepi}[1]{\overset{ #1 }{\epi}}
\newcommand{\monic}{\rightarrowtail}
\newcommand{\overmonic}[1]{\overset{ #1 }{\monic}}
\newcommand{\Integ}{\ensuremath{\mathbb{Z}}}
\newcommand{\Nat}{\ensuremath{\mathbb{N}}}
\newcommand{\Rat}{\ensuremath{\mathbb{Q}}}
\newcommand{\Comp}{\ensuremath{\mathbb{C}}}
\newcommand{\Reals}{\ensuremath{\mathbb{R}}}
\newcommand{\aff}{\ensuremath{\mathbb{A}}}
\newcommand{\proj}{\ensuremath{\mathbb{P}}}
\newcommand{\bk}{{\ensuremath{\rm \bf k}}}
\newcommand{\ck}{{\bar{\bk}}}
\newcommand{\kk}[1]{\bk^{[#1]}}

\newcommand{\Aeul}{\EuScript{A}}
\newcommand{\Beul}{\EuScript{B}}
\newcommand{\Ceul}{\EuScript{C}}
\newcommand{\Deul}{\EuScript{D}}
\newcommand{\Eeul}{\EuScript{E}}
\newcommand{\Feul}{\EuScript{F}}
\newcommand{\Geul}{\EuScript{G}}
\newcommand{\Heul}{\EuScript{H}}
\newcommand{\Keul}{\EuScript{K}}
\newcommand{\Oeul}{\EuScript{O}}
\newcommand{\Peul}{\EuScript{P}}
\newcommand{\Seul}{\EuScript{S}}

\newcommand{\Acal}{\mathcal{A}}
\newcommand{\Bcal}{\mathcal{B}}
\newcommand{\Ccal}{\mathcal{C}}
\newcommand{\Dcal}{\mathcal{D}}
\newcommand{\Ecal}{\mathcal{E}}
\newcommand{\Fcal}{\mathcal{F}}
\newcommand{\Gcal}{\mathcal{G}}
\newcommand{\OV}{\mathcal{O}}

\newcommand{\pgoth}{\mathfrak{p}}
\newcommand{\Pgoth}{\mathfrak{P}}
\newcommand{\qgoth}{\mathfrak{q}}
\newcommand{\Qgoth}{\mathfrak{Q}}
\newcommand{\m}{\mathfrak{m}}
\newcommand{\M}{\mathfrak{M}}

%%% DEAL With these later
\newcommand{\Q}{\mathbb{Q}}
\newcommand{\R}{\mathbb{R}}
\newcommand{\Z}{\mathbb{Z}}

% Unique to this doc 
\setlength\parindent{0pt} % no indent 
\usepackage{algorithmicx,algorithm,algpseudocode} % Algorithms package 
\usepackage{minted} % Embedded python 
\newcommand{\mf}{\mathfrak} % Groups, divisors,...
\newcommand{\lgr}[2]{\left(\frac{#1}{#2}\right)} % Legendre symbols
\newcommand{\rin}{\stackrel{r}{\in}} % random in set 
\newcommand{\tab}[1]{\hspace{.2\textwidth}\rlap{#1}} % Tabs in aglorithm 
\usepackage{mathtools} % Arros above 

% Tittle page 
\newcommand*{\titleGP}{
  \begingroup % Create the command for including the title page in the document
    \centering % Center all text
    \vspace*{\baselineskip} % White space at the top of the page
    \rule{\textwidth}{1.6pt}\vspace*{-\baselineskip}\vspace*{2pt} % Thick horizontal line
    \rule{\textwidth}{0.4pt}\\[\baselineskip] % Thin horizontal line
    {\huge Discrete Logarithm Based Cryptography with Abelian Varieties}\\[0.2\baselineskip] % Title
    \rule{\textwidth}{0.4pt}\vspace*{-\baselineskip}\vspace{3.2pt} % Thin horizontal line
    \rule{\textwidth}{1.6pt}\\[\baselineskip] % Thick horizontal line
    \scshape % Small caps
    A comparison of two modern techniques: \\ % Tagline(s) or further description
    The general number field sieve and \\
    Shor's quantum algorithm  \\[\baselineskip] % Tagline(s) or further description
    \today \par % Location and year
    \vspace*{2\baselineskip} % Whitespace between location/year and editors
    Prepared in fulfillment of an undergraduate honours project  \\
    by \\[\baselineskip]
    {\Large Kevin Johnson \par} % Editor list
    \vspace*{2\baselineskip} 
    {\itshape The University of Ottawa \\ Faculty of Engineering \par} % Editor affiliation
    \vspace*{3\baselineskip} 
    \begin{abstract}
      We do some stuff. 
    \end{abstract}
    \vfill
  \endgroup
}

\begin{document}
  % Front matter
  \titleGP
  \newpage
  \tableofcontents
  \newpage

  % These sections are included 
    %TODO
%References 
%Describe curves in general

% In recent years there has been increased interest in public key cryptography . The use of elliptic curves in cryptography was suggested independently by {\color{red} Neal Koblitz and Victor S. Miller in 1985 and have entered wide use in 2004 to 2005}. In this report we give explicit descriptions for algorithms need to implement public key cryptosystems using abelian varieties. 
  \section{The Discrete Logarithm Problem}
    % TODO
% Describe DSA
% Mension attacks 

Given an arbitrary finite cyclic group $G$ with group operation $\cdot$ and generator $g \in G$, discrete exponentiation by $a$ in $G$ is defined by 

$$
g^a = \overbrace{g \cdot g \cdots g}^{a \text{ times}}
$$

If $y = g^a$ is known, computing $a$ is called finding the discrete logarithm of $y$. With the method of fast exponentiation, $y$ can be computed quickly, only $O(\text{log } a)$ group operations. On the other hand, computing $a$ can be much harder. In fact, in [\ref{VictorShoup}] it was shown that in an arbitrary group for which only the group operation and discrete exponentiation can be applied to group elements, computing discrete logarithms will take at least $O(\sqrt{|G|})$ operations. In most cases though, more structure is known about the group in use.  

\subsection{The Diffie-Hellman Key Exchange}

This one-way property of discrete exponention has proven to be very useful for cryptographic purposes. The most notable of these is in the Diffie-Hellman Key Exchange Protocal in which two parties $A$ and $B$ wish to share a secret key $k$.

\begin{algorithm} 
	\caption{Diffie-Hellman Key Exchange Protocal}
	\begin{algorithmic}[1]
		\State $A$ and $B$ share a publicly known group $G$ and generator $g$.
		\State $A$ chooses a random private exponent $a$ and computes $g^a$.
		\State $B$ chooses a random private exponent $b$ and computes $g^b$.
		\State $A$ sends $g^a$ to $B$ and $B$ sends $g^b$ to $A$. 
		\State $A$ raises $g^b$ to their own private exponent $a$ to obtain $k = (g^b)^a = g^{ab}$.
		\State $B$ raises $g^a$ to their own private exponent $b$ to obtain $k = (g^a)^b = g^{ba}$.
	\end{algorithmic} 
\end{algorithm}  


The two parties may now use $k$ to communicate with any cryptographically secure communication protocol needing a symetric key. The described protocol relies on the hardness of computing $g^{ab}$ given $g^a$ and $g^b$, which is conjectured in [\ref{WhitfieldDiffieMartinHellman}] to be equivalent to computing discrete logarithms. This protocal has the nice property that it's applicable to any finite cyclic group. We look at some simple examples of groups which have already had significant use in cryptography.  

\subsection{A Brief History of the Groups $\mathbb{F}_p^*$ and $\mathbb{F}_{2^n}^*$}

Given a finite field $\mathbb{F}$, the multiplicative units $\mathbb{F}^* = \mathbb{F} \backslash \lbrace 0 \rbrace $ form a finite cyclic group and thus may be used for discrete logarithm based cryptography. It is a standard result from algebra that every finite field has order $p^n$, where $p$ is a prime and $n \in \Z^+$. We divide the discussion of the cryptographic properties of the group $\mathbb{F}_{p^n}^*$ into two cases; when $n = 1$ and when $n>1$. \\

In the latter case, when working with finite fields of order $p^n$, the arithmetic is only really tractable when $p=2$. In the 1980's, researchers at the University of Waterloo made attempts to construct discrete logarithm cryptosystems based on $\mathbb{F}_{2^{127}}^*$ which initially paralelled RSA in terms of bits of security. But in 1986, Don Coppersmith devised an astonishing algorithm in [\ref{DanCoppersmith}] which could compute discrete logarithms in the group $\mathbb{F}_{2^{127}}^*$ in about $5$ minutes. Further attempts were made to increase the size to $n=593$ but similar adaptations of Coppersmith's algorithm made researchers abandon public key cryptosystems based on the discrete logarithm problem in $\mathbb{F}_{2^n}^*$. \\

When $n = 1$, we have a group which is essentially just the non-zero integers mod a prime. As one might expect, when $p$ is small, computing discrete logs in $\mathbb{F}_p^*$ can be done quickly with just trial exponentiation. When $p$ is large though, say $p = 2^{1000}$, this method becomes completely intractable, even on todays fastest computers. That being said, there is an attack described in [\ref{NFStoDLP}] which adapts the Number Field Sieve to solve discrete logs in $\mathbb{F}_p^*$. This attack has running time very similar to factoring $$O(p) = e^{1.923(\log p)^{1/3}(\log\log p)^{2/3}}$$ This basically means that the bit length required for $p$ in $\mathbb{F}_p^*$ based cryptosystems is the same as the bit length required for the modulus in RSA. In todays standards that means $p = 2^{2048}$. Although this is cryptographically viable, in practice using such large values of $p$ has its limitations, such as bandwidth in network communications or memory in a hand-held devices. These two cases made researchers search for alternative groups with cryptographically strong properties. 









  \section{Abelian Varieties}
    %TODO
%Proper definition of algebraic variety
%Definition of Dimension 
%Definition of Genus
%Start with algebraically closed field


% there should be some mention of k rational points

In this report we focus on groups which arise from the solution set of polynomial equations over finite fields. The most famous of these groups is the set of points on an elliptic curve characterized by the equation $y^2 = x^3 + ax + b$ where $a,b \in F_p$. In recent years, this group has found significant success in public key cryptography even though it's an open questions whether this group is actually cryptographically secure. The main benefit of using elliptic curves is that the fastest known attack on them has $O(\sqrt{N})$ complexity, where $N$ is the order of the group. This means significantly smaller keys can be used in comparison to the key length of RSA. A natural question is 

$$
\textbf{What about other polynomial equations?}
$$ 

It turns out that there are infinitely many groups which arise from the solution sets of polynomial equations. Which ones make crpytographically strong groups is a active area of research. First we develope some theory required to speak about these groups. \\ 

When working with polynomials in more than one variable, it is sometimes simpler to use the formalism of classic algebraic geometry as found in [\ref{Hartshorne}]. Let $K$ be a field (for the purposes of this report we will only be interested in $K = \mathbb{F}_p$, where $p>3$). Let $A = K[X_1,...,X_n]$ represent the polynomial ring in $n$ variables over $K$. Given a subset $T \subseteq A $, we may define 

$$
Z(T) = \lbrace P \in K^n \mid f(P) = 0 \text{ for all } f \in T \rbrace 
$$ 

to be the set of all common zeros of polynomials in $T$. We call $Y \subseteq K^n $ an \textit{algebraic subset} if $Y = Z(T)$ for some subset $T \subseteq A$. Similary, given an subset $Y \subseteq K^n$, may define the set of A 

$$ 
I(Y) = \lbrace f \in A \mid f(P) = 0 \text { for all } P \in Y \rbrace 
$$ 

In fact, $I(Y)$ is an ideal in $A$ often called \textit{the ideal of Y}. So we may consider the quotient ring 

$$
A(Y) = A / I(Y) 
$$ 

We refer to this ring as the \textit{the coordinate ring of Y}. This ring encodes a lot of information about an alebraic set. 



\subsection{Dimension}
\subsection{Genus}



  \section{Elliptic Curves}
    %TODO
%Finish finding point algorithm
%Pictures for group operation 
%Give algorithm for adding points 
%Generating provably random points


Let $p$ be an odd prime and let $E = Z( y^2 - x^3 - ax - b)$. We call $E$ and \textit{elliptic curve} defined over $K = \mathbb{Z}_p$ if $4a^3 + 27b^2 \not\equiv 0 \text{ mod } p$. It was first realized by {\color{red}Need reference} Abel and Jacobi in the 1700's that remarkably, the $\mathbb{Z}_p$-rational points of $E$ can be transformed into a group using a very specific group operation. In this section we describe this group operation along with other algorithms needed for cryptographic purposes. 

\subsection{The Group Operation}  

Given two $\mathbb{Z}_p$-rational points $P_1 = (x_1,y_1),P_2=(x_2,x_2)$ which we want to add together, we first define the value 

$$ s =
\begin{cases}
\frac{y_1 - y_2}{x_1 - x_2} \text{ mod } p 	&\text{if } P_1 \neq P_2 \\
\frac{3x_1^2 - a}{2y_1} 	\text{ mod } p	&\text{if } P_1 = P_2 
\end{cases} 
$$ 

Then we define the coordinates of the point $P_3 = P_1 + P_2 $ to be 

\begin{align*}
	x_3 &= s^2 - x_1 - x_2  \\ 
	y_3 &= y_1 + s(x_3 - x_1) 
\end{align*}

It's not immediately obvious that $P_3 = (x_3,y_3)$ is even a point on $E$ and less obvious that this operation satisfies the axioms of a group.  \\

\begin{algorithm} 
	\caption{The addition of two points $P_1 = (x_1,y_1),P_2 = (x_2,y_2)$ on an elliptic curve $E : y^2 - x^3 - ax - b$}
	\begin{algorithmic}[1]
		\Function{Add}{$E$,$P_1,P_2$}
			\If{$P_1 = \mathcal{O}$} 
				\State \Return{$P_2$}
			\ElsIf{$P_2 = \mathcal{O}$}
				\State \Return{$P_1$}
			\ElsIf{$P_1 = P_1$}
				\State $ s \leftarrow (3x_1^2 - a)(2y_1)^{-1} \text { mod } p $
		  	\Else
	  			\If{$x_1 \neq x_2$} 
	  				\State $ s \leftarrow (y_1 - y_2)(x_1 - x_2)^{-1} \text{ mod }p $
	  			\Else
	  				\State \Return{$\mathcal{O}$}
	  			\EndIf
	  		\EndIf 
	  		\State $ x_3 \leftarrow s^2 - x_1 -x_2 \text{ mod } p $
	  		\State $ y_3 \leftarrow  y_1 + s(x_3 - x_1) \text{ mod } p $ 
	  		\State \Return{$P_3 = (x_3,y_3)$}
	  	\EndFunction
	\end{algorithmic} 
\end{algorithm} 

An implementation note is that in steps $7$ and $10$, modular inverses must be calculated. Also note that when $x_1 = x_2$ but $P_1 \neq P_2$, the second coordinates satisfy $y_1 = -y_2$. So the line from $P_1$ to $P_2$ is just a vertical line at $x_1$. This is taken to be the point at infinity $\mathcal{O}$.

\subsection{Scalar Multiplication of a Point}

For many discrete logarithm protocols (such as Diffie-Hellman of DSA), we require to add point $P$ to itself many times in order to perform discrete exponentiation. That is, given an integer $m$ we need to calculate $$mP = \overbrace{P + P + \cdots + P}{m \text{ times}}$$ fast in order to be cryptographically reasonable. The following algorithm does this.

\begin{algorithm} 
	\caption{Scalar multiplication of a point $P$ by an integer $m$}
	\begin{algorithmic}[1]
		\Function{ScalarMult}{$m$,$P$}
		  	\If{$m = 0 $}
		  		\State \Return{$\mathcal{O}$}
		  	\ElsIf{$m = 1 $}
		  		\State \Return{$P$}
		  	\ElsIf{$m \equiv 0 \text{ mod } 2$}
		  		\State \Return{\Call{ScalarMult}{$m/2$,$P + P$}}
		  	\Else 
		  		\State \Return{$P$ + \Call{ScalarMult}{$m-1$,$P$}}
		  	\EndIf
	  	\EndFunction
	\end{algorithmic} 
\end{algorithm} 

\subsection{Finding Points}

Now that we have developed arithemtic on and elliptic curve $E$, then next step is figure out how to find points on $E$. If $y^2 = x^3 + ax + b$ for $a,b \in \mathbb{Z}_p$, then finding $\mathbb{Z}_p$-rational points on $E$ is equivalent to determining if $x^3 + ax + b$ is a square mod $p$. This is a classical problem in number theory which can be reformulated to determing the value of the \textit{Legendre Symbol}, which is defined as 

$$ \lgr{a}{p} =
\begin{cases}
1 &\text{if }a \text{ is a square mod } p\\
-1 &\text{if }a \text{ is not a square mod }p\\
0 & \text{if } p \text{ divides }a
\end{cases} 
$$ 

where (in our case) $a = x^3 + ax + b$. The following five properties let us determine whether $a$ is a square mod $p$ in polynomial time. Let $a,b \in \mathbb{Z}$ and $p,q$ be odds primes.

\begin{enumerate}[(i)]
	\item \label{modequiv} If $a \equiv b$ mod $p$, then $\lgr{a}{p} = \lgr{b}{p}$
	\item \label{multiplicative} $\lgr{ab}{p} = \lgr{a}{p} \lgr{a}{p} $
	\item \label{1iseasy} $\lgr{-1}{p} = 1$ if $p \equiv 1 $ mod $4$, and $\lgr{-1}{p} = -1$ if $p \equiv 3 $ mod $4$
	\item \label{2iseasy} $\lgr{2}{p} = 1$ if $p \equiv \pm 1 $ mod $8$, and $\lgr{2}{p} = -1$ if $p \equiv \pm 3 $ mod $8$
	\item \label{reciprocity} If $p,q$ are distinct, then 
		$$ 
			\lgr{p}{q} = \lgr{q}{p} \text { if } p \text { or } q \equiv 1 \text{ mod } 4
		$$ 
		$$  
			\lgr{p}{q} = - \lgr{q}{p} \text { if } p \equiv q \equiv 3 \text{ mod } 4
		$$
\end{enumerate}

For example, to determine if $105$ is a square mod $227$, we simply compute 

\begin{align*}
	\lgr{105}{227} &\stackrel{\text{\eqref{multiplicative}}}{=} \lgr{3}{227}\lgr{5}{227}\lgr{7}{227} \\
	&\stackrel{\text{\eqref{reciprocity}}}{=} (-1) \lgr{227}{3}\lgr{227}{5} (-1) \lgr{227}{7} \\
	&\stackrel{\text{\eqref{modequiv}}}{=} \lgr{2}{3}\lgr{2}{5}\lgr{3}{7} \\
	&\stackrel{\text{\eqref{2iseasy}}}{=} (-1)(-1)(-1) \lgr{7}{3} \\
	&\stackrel{\text{\eqref{modequiv}}}{=} (-1) \lgr{1}{3} \\
	&= -1 
\end{align*}

So $105$ is a not a square mod $227$. A full description of how to compute Legendre symbols is provided in algorithm  \ref{legendresymbol}. This process is incrediably quick but doesn't actually tell us the squareroot of $a$, if $a$ is indeed a square mod $p$. If $\lgr{a}{p} = 1 $, the following method finds $x$ such that $x^2 = a$ mod $p$. We break the calcultions into two cases. \\

If $p \equiv 3 $ mod $4$, then $x = a^{(p+1)/4}$ satisfies $x^2 \equiv a $ mod $p$. The second case where $p \equiv 1 $ mod $4$ is more involved.

\begin{enumerate}[1.]
	\item Pick random $r$ such that $\lgr{r^2 - 4a}{p} =  -1 $ and write $d = r^2 - 4a$.
	\item let $\alpha = \frac{r+\sqrt{d}}{2}$ and $\alpha^k = \frac{V_k + U_k \sqrt{d}}{2}$ where $V_k,U_k$ are the coefficients of $1, \sqrt{d}$ in the $k$-th power of $\alpha$. 
	\item $x =  2^{-1}V_{(p+1)/2}$ satisfies $x^2 \equiv a $ mod $p$. 
\end{enumerate}

For this case, the proof that $x$ does satisfy $x^2 \equiv a $ mod $p$ is quite involved but can be found in {\color{red} GarrWALSH NOTES}. Putting all this together we obtain the following algorthm which finds random points on an elliptic curve $E$. \\ 

\begin{algorithm} 
	\caption{Find random points on elliptic curve $E : y^2 - x^3 - ax - b$ modulo an odd prime $p$ }
	\begin{algorithmic}[1]
		\Function{RandomPoints}{$E$,$p$}
			\State pick random $x \in \lbrace 1, \dots , p-1 \rbrace $
			\While{\textbf{not} \Call{Legendre}{$x^3 + ax + b$,$p$}}
				\State pick another $x \in  \lbrace 1, \dots , p-1 \rbrace $
			\EndWhile
	  	\EndFunction
	\end{algorithmic} 
\end{algorithm} 


\subsection{Counting Points}

When using an elliptic curve $E$ for discrete logarithm based cryptosystems, it is importance to know the number of points on $E$. This is because in 1978 Stephan Pohlig and Martin Hellman came up with an attack, described in [\ref{PohligHellman}], which uses the order of the group to solve discrete logs. The attack proceeds as follows: Given $G$, a finite cyclic group with order $N$, we may factor $N = p_1^{e_1}p_2^{e_2} \cdots p_s^{e_s}$ where $p_1,p_2,...,p_s$ are primes and $e_1,e_2,...,e_s \in \Z^+ $. All the subgroups $G_1,G_2,...,G_s$ of $G$ will have order $p_1^{e_1},p_2^{e_2},...,p_s^{e_s}$ respectively. Given an element $h = g^a \in G$, the Pohlig-Hellman attack solves for $a$ in the smaller subgroups $G_1,G_2,...,G_s$ and uses the chinese remainder theorem to piece these solutions back together to solve for $a$ in the bigger group $G$. What they noticed is that when using this method, the complexity of solving for $a$ in $G$ (which can be done in $O(\sqrt{N})$ bit operations) gets reduced to $O(p_1^{e_1/2}) + O(p_2^{e_2/2}) + O(p_s^{e_s/2})$, i.e. the complexity the sum of the complexities of solving in the smaller groups. This means a group's bits of security is only as high as the bits of security of its largest subgroup. Therefore we want the order of the groups that we use to be prime, or at least a very small multiple of a prime, to avoid this attack. \\

With this is mind, we need an algorithm which calculates the number of points on an elliptic curve $E$ and thus the order of the group $E$. Intuitively, the number of points must be much less than $p^2$, for this would would correspond to a curve that zeros every point in $\Z_p^2$. In fact, a much better estimate is given by Hasse in [\ref{hasseBound}]. Denote the number of $\mathbb{F}_p$-points on $E$ by $\# E(\mathbb{F}_p)$, then  

\begin{equation}
	\# E(\mathbb{F}_p) = q + 1 + t
\end{equation}

where $|t| \leq 2  \sqrt{q}$. So essentially finding $\# E(\mathbb{F}_p)$ is equivalent to finding $t$. The first tractable algorithm was designed by Rene Schoof in 1985 and was inspired by a very special map called \textit{the map of frobeneous}
\begin{align*}
	\phi : E &\rightarrow E \\
	(x,y) &\mapsto (x^p,y^p)
\end{align*}
which satisfies the equation 

\begin{equation} 
	\phi^2 + p = - t \phi \label{characteristic}
\end{equation} 

 in the endomorphism ring $\text{End}(E)$. The idea is to find $t$ in the interval $[-2 \sqrt{p},2 \sqrt{p}]$ which gives equivalnce in \eqref{characteristic}. The problem is when $p$ is large ($300$ bits), this interval is too big to brute force. Schoof's idea was to reduce this guess work by computing $t$ which satisfies \eqref{characteristic} in the $l$-torsion subgroups $E[l]$ for a sufficient number of small primes $l$, then use the chinese remainder theorem to recover $t$. 


\begin{algorithm}
	\caption{Schoofs algorithm to calculate the number of points on elliptic curve $E : y^2 - x^3 - ax - b$ over $\mathbb{F}_p$}
	\begin{algorithmic}[1]
		\Function{Order}{$E$,$p$}
			\State $S \leftarrow $ set of odd primes such that $\prod_{l \in S}l > 4\sqrt{p}$  
			\State $C \leftarrow \emptyset $
			\If{$\text{gcd}(x^p - x, x^3 + ax + b) = 1$}
				\State $C \leftarrow C \cup \lbrace (2,1) \rbrace $
			\Else 
				\State $C \leftarrow C \cup \lbrace (2,0) \rbrace $
			\EndIf

			\For{$l \in S$}
				\State $\psi_l \leftarrow $ \Call{DivisionPolynomial}{$l$} 
				\State $p_l = p \text{ mod } l$
				\State $(x^\prime,y^\prime) \leftarrow (x^{p^2},y^{p^2}) \oplus [p_l](x,y) $ 
				\For {$t \in \lbrace 1,2, ..., \frac{l-1}{2} \rbrace $}
					\State $(x_{t},y_{t}) = [t](x,y)$
					\If{$x^\prime == x_t \text{ mod } \psi_l$}
						\If{$\frac{(y^\prime - y_t)}{y} == 0 \text{ mod } \psi_l$}
							\State $C \leftarrow C \cup \lbrace (l,t) \rbrace $
						\Else
							\State $C \leftarrow C \cup \lbrace (l,-t) \rbrace $
						\EndIf
					\EndIf
				\EndFor 
			\EndFor 
	  	\EndFunction
	\end{algorithmic} 
\end{algorithm} 





\subsection{Generating Provably Random Elliptic Curves}

  \section{Hyperelliptic Curves}
    Unlike elliptic curves, when the genus $\mf{g}$ of a curve $\mf{C} $ is greater than $1$, the set of points on $\mf{C}$ will not always form a group. 

\begin{example}

\end{example}

\subsection{The Jacobian of a Hyperelliptic Curve}

Luckily, there is another way to form an abelian group with hyperelliptic curves. Indeed, let $\mf{D}$ be the set of all formal finite sums 

$$ \sum_i m_i P_i $$ 

where $m_i \in \mathbb{Z}$ and $P_i$ are points on the curve $\mf{C}$. We call elements of $\mf{D}$ divisors of $\mf{C}$. Given a rational function $f$ in $\mathbb{Z}_p[\mf{C}]$, we can define the corresponding divisor to $f$ as

$$(f) = \sum_i m_i P_i $$ where $P_i$ are the zeros and poles of $f$ with multiplicities $m_i$. Divisors of this form are called principal divisors and we let $\mf{P}$ denote the subset of all of them in $\mf{D}$. If we define the operation on $\mf{D}$ by 

$$ \sum_i m_i P_i  + \sum_i m^\prime_i P_i  = \sum_i (m_i+m^\prime) P_i $$ 

then $\mf{D}$ becomes and abelian group. Unfortunetly, this group is far too large for cryptographic purposes. So we consider the subgroup $\mf{D}^0$ of all divisors of $\mf{D}$ whos coefficients sum to $0$. That is, divisors $ \sum_i m_i P_i $ such that $\sum_i m_i = 0$. \\ 

Even though this subgroup is still infinite, we can define two divisors $D_1, D_2$ of $\mf{D}^0$ to be equal if $D_1 - D_2$ is equal to the divisor of a rational function on $\mf{C}$. That is, $D_1 - D_2 = (f) $ for $f \in \mathbb{Z}_p[\mf{C}]$. This new quotient group, denoted $$ \mf{J} = \mf{D}^0 / \mf{P}$$ is called the jacobian of the curve $\mf{C}$ and is a finite abelian group. This will be the group used to build hyperelliptic cryptosystems.

\subsection{Representation of Divisors}

Athough the Jacobian $\mf{J}$ of an hyperelliptic curve $\mf{C}$ is a finite abelian group, elements of $\mf{J}$ are very hard to represent. 

\begin{example}
\end{example} 


  \section{Algorithms}
    \begin{algorithm} \label{legendresymbol} 
	\caption{The Legendre symbol of an integer $a$ modulo a prime $p$}
	\begin{algorithmic}[1]
		\Function{Legendre}{$a$,$p$}
			\State $s \leftarrow 1 $
			\For{$q \in  \text{FACTORS}(a)$}
				\State $s \leftarrow s * $\Call{LegendreForPrime}{$q$,$p$}
			\EndFor
			\Function{LegendreForPrime}{$q$,$p$} 
			  	\If{$q =1$}
			  		\State \Return $1$
			  	\ElsIf{$q \equiv 0 \text{ mod } p$}
			  		\State \Return $0$ 
			  	\ElsIf{$q = -1 $}
			  		\If{$p \equiv 1 \text{ mod } 4$}
			  			\State \Return $1$
			  		\Else
			  			\State \Return $-1$
			  		\EndIf
			  	\ElsIf{$q = 2$}
			  		\If{$p \equiv \pm 1 \text{ mod } 8$}
			  			\State \Return $1$
			  		\Else
			  			\State \Return $-1$
			  		\EndIf 
			  	\ElsIf{$q > p$}
			  		\State \Return \Call{LegendreForPrime}{$q \text{ mod } p$,$p$} 
			  	\ElsIf{$q \equiv 1 \text{ mod } 4$ or $p \equiv 1 \text{ mod } 4$}
			  		\State \Return \Call{LegendreForPrime}{$p$,$q$}
			  	\Else
			  		\State \Return $-$\Call{LegendreForPrime}{$p$,$q$}
			  	\EndIf
		  	\EndFunction
		  	\State \Return{$s$}
		\EndFunction
	\end{algorithmic} 
\end{algorithm} 

\begin{algorithm} \label{schoof}
	\caption{Schoofs algorithm to calculate the number of points on elliptic curve $\mathbb{E} : y^2 - x^3 - ax - b$ over $\mathbb{F}_p$}
	\begin{algorithmic}[1]
		\Function{Order}{$\mathbb{E}$,$p$}
			\State $S \leftarrow $ set of odd primes such that $\prod_{l \in S}l > 4\sqrt{p}$  
			\State $C \leftarrow \emptyset $
			\If{$\text{gcd}(x^p - x, x^3 + ax + b) = 1$}
				\State $C \leftarrow C \cup \lbrace (2,1) \rbrace $
			\Else 
				\State $C \leftarrow C \cup \lbrace (2,0) \rbrace $
			\EndIf

			\For{$l \in S$}
				\State $\psi_l \leftarrow $ \Call{DivisionPolynomial}{$l$} 
				\State $p_l = p \text{ mod } l$
				\State $(x^\prime,y^\prime) \leftarrow (x^{p^2},y^{p^2}) \oplus [p_l](x,y) $ 
				\For {$t \in \lbrace 1,2, ..., \frac{l-1}{2} \rbrace $}
					\State $(x_{t},y_{t}) = [t](x,y)$
					\If{$x^\prime = x_t \text{ mod } \psi_l$}
						\If{$\frac{(y^\prime - y_t)}{y} = 0 \text{ mod } \psi_l$}
							\State $C \leftarrow C \cup \lbrace (l,t) \rbrace $
						\Else
							\State $C \leftarrow C \cup \lbrace (l,-t) \rbrace $
						\EndIf
						\State Next $l$
					\EndIf
				\EndFor 
				\If{$p = w^2 \text{ mod } l$ for some $w\in Z$}
					\If{$\text{gcd}(\text{numerator}(x^p - x_w),\psi_l) = 1$}
						\State $C \leftarrow C \cup \lbrace (l,0) \rbrace $
					\Else
						\If{$\text{gcd}(\text{numerator}(\frac{y^p - y_w}{y}),\psi_l) = 1$}
							\State $C \leftarrow C \cup \lbrace (l,-2w) \rbrace $
						\Else
							\State $C \leftarrow C \cup \lbrace (l,2w) \rbrace $
						\EndIf 
					\EndIf
				\Else
					\State $C \leftarrow C \cup \lbrace (l,0) \rbrace $
				\EndIf 
			\EndFor 
			\State \Return{\Call{ChineseRemainderTheorem}{C}} 
	  	\EndFunction
	\end{algorithmic} 
\end{algorithm} 


    \newpage
  \section{References}
    

% \item \label{} Author, \textit{Tittle}, Journal, Date, page 


\begin{enumerate}[1]

	\item \label{Hartshorne} Robin Hartshorne, \textit{Algebraic Geometry, Graduate Texts in Mathematics, vol. 52}, Springer-Verlag, New York, 1977, ISBN 0-387-90244-9.   
	
	\item \label{VictorShoup} Victor Shoup, \textit{Lower bounds for discrete logarithms and related problems}, Theory and Application of Cryptographic Techniques, 1997, pp. 256 - 266. 
	
	\item \label{WhitfieldDiffieMartinHellman} Whitfield Diffie and Martin E. Hellman, \textit{New directions in cryptography}, IEEE Transactions on Information Theory \textbf{IT-22} (1976), no. 644-654.  
\end{enumerate}
\end{document}




